\MPsecnum{7}
\MPentry{SynTab}
\MPsection{BEZEICHNUNG}
\MPcaption{Werkzeuge zur Tabellengenerierung}
\MPsection{�BERSICHT}
\begin{MPhangpar}
{\tt{}\#include $<$syntab.tm$>$}\end{MPhangpar}

\begin{MPhangpar}
{\tt{}SynTab SynTab (IO {\it{}io}, bool {\it{}clrpage} = true)}\end{MPhangpar}

\begin{MPhangpar}
{\tt{}SynTab SynTab (IO {\it{}io}, int {\it{}margin}, bool {\it{}clrpage} = true)}\end{MPhangpar}

\begin{MPhangpar}
{\tt{}void SynTab::printf (str {\it{}fmt}, ...)}\end{MPhangpar}

\begin{MPhangpar}
{\tt{}void SynTab::line (str {\it{}text})}\end{MPhangpar}

\begin{MPhangpar}
{\tt{}void SynTab::caption (str {\it{}hname}, str {\it{}sname})}\end{MPhangpar}

\begin{MPhangpar}
{\tt{}void SynTab::def (str {\it{}name}, str {\it{}def})}\end{MPhangpar}


\begin{MPhangpar}
{\tt{}virtual void SynTab::note (str {\it{}name})}\end{MPhangpar}

\begin{MPhangpar}
{\tt{}virtual void SynTab::note (...)}\end{MPhangpar}

\begin{MPhangpar}
{\tt{}void SynTab::begin (TabLabel {\it{}col}, int {\it{}width} = 2, int {\it{}height} = 2)}\end{MPhangpar}

\begin{MPhangpar}
{\tt{}void SynTab::body (TabLabel {\it{}line}, mdmat {\it{}md}, VirFunc {\it{}idx} = NULL, int {\it{}lag} = 0)}\end{MPhangpar}

\begin{MPhangpar}
{\tt{}virtual void SynTab::tabline (str {\it{}label})}\end{MPhangpar}

\begin{MPhangpar}
{\tt{}virtual void SynTab::tabline (str {\it{}label}, ...)}\end{MPhangpar}

\begin{MPhangpar}
{\tt{}void SynTab::end ()}\end{MPhangpar}

\MPsection{BESCHREIBUNG}
Der Datentype {\tt{}SynTab} dient zur Generierung
einer Tabelle im Synthesis{-}Standard. Er dient zur Konstruktion einer
{\tt{}SynTab}{-}Umgebung, die in  eine {\tt{}SynTable}{-}Umgebung eingebettet ist.

Eine {\tt{}SynTab}{-}Struktur wird mit einem Konstruktor
eingerichtet, der automatisch auch den Beginn der zugeh�rigen
{\tt{}SynTable}{-}Umgebung generiert.

Bei den Konstruktoren gibt
die Variable {\it{}io} die aktuelle Ausgabestruktur an.
Wahlweise kann eine Tabelle mit der aktuellen Randeinstellung oder
mit expliziter Angabe des Randes generiert werden.
Falls das optionale Flag {\it{}clrpage} gesetzt ist,
wird die Tabelle auf einer
neuen Seite begonnen.

Der Abschlu� der {\tt{}SynTable}{-}Umgebung erfolgt automatisch bei Freigabe
der {\tt{}SynTab}{-}Struktur. Eine explizite Freigabe ist mit der
Funktion {\tt{}free()}, oder durch Initialisierung mit NULL m�glich.
Endet die Lebensdauer einer {\tt{}SynTab}{-}Struktur nach einer
Tabellendefinition, kann auf die explizite Freigabe verzichtet werden,
da der Destruktor automatisch aufgerufen wird.

Die Tabellen�berschrift wird am einfachsten mit der Funktion
{\tt{}SynTab::caption} generiert.
Als Argument wird die Haupt{-} und Unter�berschrift �bergeben.
Sie erzeugt die Ausgabe
{\tt{}$\backslash$caption\{{\it{}hname}\}} und {\tt{}$\backslash$subcaption\{{\it{}sname}\}}.

Tabellenanmerkungen k�nnen mit der Funktion {\tt{}note} generiert
werden. Dabei kann entweder ein einzelner String (f�r eine Anmerkung)
oder eine Stringliste mit den Namen von Anmerkungen �bergeben
werden. Dabei wird angenommen, da� entsprechende Anmerkungen mit
den \LaTeX{-}Macros {\tt{}$\backslash$setnote} (Synthesis{-}Standard) eingerichtet wurden.

Zur Konstruktion beliebiger Textteile in der Tabellenumgebung stehen noch
folgende Funktionen zur Verf�gung:

\begin{MPlist}
\MPitem{{\tt{}SynTab::printf}}
Erlaubt eine formatierte Ausgabe von Variablen
(Vergleiche dazu {\it{}printf(3)}).

\MPitem{{\tt{}SynTab::line}}
gibt {\it{}text}, gefolgt von einem Zeilenvorschub aus.

\MPitem{{\tt{}SynTab::def}}
dient zur Definition einfacher \LaTeX{-}Makros mit dem Namen
{\it{}name} und dem Definitionsstring {\it{}def}.

\end{MPlist}
\MPsubsection{Tabellenkonstruktion}

Zur Konstruktion von Tabellen stehen die Funktionen {\tt{}SynTab::begin},
{\tt{}SynTab::body} und {\tt{}SynTab::end} zur Verf�gung.

Mit {\tt{}SynTab::begin} wird der Tabellenkopf eingeleitet.
Als Argument wird der Achsenbezeichner f�r die Spalten
und optional die Standardbreite
einer Spalte und die Standardh�he einer Zeile �bergeben.

Der Tabelenk�rper wird mit der Funktion {\tt{}SynTab::body}
generiert. Als erstes Argument wird die Achsenstruktur f�r
die Tabellenzeilen �bergeben. Danach folgt die Datenmatrix
mit den Tabellendaten und die zugeh�rigen Auswertungsparameter.

Die Tabellendaten werden mit der Funktion {\tt{}mdmat::data} generiert,
wobei der Definitionsstring von der Achsenstruktur {\it{}line} bestimmt wird.
Die anderen Parameter werden von der Funktion {\tt{}SynTab::body} �bernommen.

Eine einzelne Tabellenzeile wird mit der Funktion
\MPitem{{\tt{}SynTab::tabline}} generiert. Als Argument wird der
Zeilenbezeichner {\it{}label} und die Liste der Tabellenwerte
ausgegeben. Fehlen die Tabellenwerte, wird eine Leerzeile (mit
h�ngenden Linien) generiert.

Die Funktion {\tt{}SynTab::end} schlie�t eine Tabelle ab.

\MPsection{BEISPIEL}
Das folgende Beispiel
{\tt{}\begin{MPnofill}
\#include $<$syntab.tm$>$
~
/$\ast$      Testdaten und Achsenbezeichner
$\ast$/
~
mdmat md = mdread ((IO) string !
2       c1      c2      c3
l1      1.1     1.2     1.3
l2      2.1     2.2     2.3
l3      3.1     3.2     3.3
!, {\tt\symbol{34}}type=double x=Spalten y=Zeilen{\tt\symbol{34}});
~
TabLabel col = TabLabel({\tt\symbol{34}}col{\tt\symbol{34}}, {\tt\symbol{34}}Spalten{\tt\symbol{34}}, string !
c1      Spalte 1
c2      Spalte 2
c3      Spalte 3
!);
~
TabLabel line = TabLabel({\tt\symbol{34}}line{\tt\symbol{34}}, {\tt\symbol{34}}Zeilen{\tt\symbol{34}}, string !
l1      Zeile 1
l2      Zeile 2
l3      Zeile 3
!);
~
/$\ast$      Tabelle
$\ast$/
~
SynTab tab = SynTab(iostd);
tab.caption({\tt\symbol{34}}Testtabelle{\tt\symbol{34}}, {\tt\symbol{34}}Untertitel{\tt\symbol{34}});
tab.begin(col);
tab.body(line, md);
tab.end();
tab.note({\tt\symbol{34}}n1{\tt\symbol{34}});
free(tab);
\end{MPnofill}
}generiert die folgende Ausgabe

{\tt{}\begin{MPnofill}
$\backslash$clearpage
$\backslash$begin\{SynTable\}
$\backslash$caption\{Testtabelle\}
$\backslash$subcaption\{Untertitel\}
~
$\backslash$begin\{SynTab\}\{2\}\{2,2,2\}
$\backslash$thislinebold
$\backslash$norule \& Spalte 1 \& Spalte 2 \& Spalte 3 $\backslash$cr
Zeile 1 \& 1.10 \& 1.20 \& 1.30 $\backslash$cr
Zeile 2 \& 2.10 \& 2.20 \& 2.30 $\backslash$cr
Zeile 3 \& 3.10 \& 3.20 \& 3.30 $\backslash$cr
$\backslash$end\{SynTab\}
~
$\backslash$note\{n1\}
$\backslash$end\{SynTable\}
\end{MPnofill}
}\MPsection{SIEHE AUCH}
printf(3) im {\it{}Programmers's Reference Manual}.\newline
texmerge(1), TabLabel(7).

\MPsection{BEMERKUNGEN}
Die Variablen {\tt{}PrintListBegin}, {\tt{}PrintListDelim} und {\tt{}PrintListEnd}
werden bei der Tabellenkonstruktion umdefiniert.
