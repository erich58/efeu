\MPsecnum{7}
\MPentry{TabLabel}
\MPsection{BEZEICHNUNG}
\MPcaption{Tabellenachsen}
\MPsection{�BERSICHT}
\begin{MPhangpar}
{\tt{}\#include $<$tablabel.tm$>$}\end{MPhangpar}

\begin{MPhangpar}
{\tt{}TabLabel TabLabel (str {\it{}key}, str {\it{}name})}\end{MPhangpar}

\begin{MPhangpar}
{\tt{}TabLabel TabLabel(str {\it{}key}, str {\it{}name}, IO {\it{}io})}\end{MPhangpar}

\begin{MPhangpar}
{\tt{}TabLabel TabLabel::entry(str {\it{}key}, str {\it{}label})}\end{MPhangpar}

\begin{MPhangpar}
{\tt{}int CompLabel(TabLabel {\it{}a}, TabLabel {\it{}b})}\end{MPhangpar}

\begin{MPhangpar}
{\tt{}TabLabel CDummyTabLabel;}\end{MPhangpar}

\begin{MPhangpar}
{\tt{}TabLabel LDummyTabLabel;}\end{MPhangpar}

\MPsection{BESCHREIBUNG}
Der Datentype {\tt{}TabLabel} definiert eine Tabellenachse.
Er wird in Kombination mit dem Datentype {\tt{}SynTab}
(Vergleiche dazu {\it{}SynTab(7)}) zur Einarbeitung von Daten
aus einer Datenmatrix verwendet.

Eine Tabellenachse kann sowohl f�r die Zeilen{-} als auch die
Spalten einer Tabelle verwendet werden.
Sie hat den folgenden Aufbau:

\begin{MPnofill}
struct TabLabel \{
~       str key;        /$\ast$ Kennung $\ast$/
~       str name;       /$\ast$ Achsenname $\ast$/
~       str list;       /$\ast$ Selektionsliste $\ast$/
~       DataBase tab;   /$\ast$ Indextabelle $\ast$/
~       str special;    /$\ast$ Sonderdefinitionen $\ast$/
\};
\end{MPnofill}
Dabei gibt {\tt{}TabLabel::key} die Kennung der Tabellenachse
und {\tt{}TabLabel::name} den
Namen der zugeh�rigen Achse in der Datenmatrix an. Die
Variable {\tt{}TabLabel::list} enth�lt die Selektionsliste mit den
Achsenelementen. Sie wird beim Laden der Datenmatrix zur
Generierung von Zusammenfassungen ben�tigt und sorgt dabei
f�r die richtigen Reihenfolge der Achsenelemente (Wichtig bei Spalten!).
Der Konverter {\tt{}str TabLabel()} liefert den Selektionsstring
der Achsenstruktur. Damit kann die Achse stellvertretend f�r
seine Komponente beim Laden einer Datenmatrix angegeben werden.

Die Datenbank {\tt{}TabLabel::tab} enth�lt die einzelen
Zeilen{-} bzw. Spaltenbezeichner und die zugeh�rigen
Selektionsdefinitionen der Datenmatrix.

In {\tt{}TabLabel::special} stehen LaTeX{-}Befehle, die zu Beginn
eines Tabellenkopfes (Spaltenachse) bzw. Tabellenk�rpers (Zeilenachse)
auszugeben sind.

\MPsubsection{Konstruktion von Achsendefinitionen}
Achsendefintionen werden am einfachsten mit dem Konstruktor
{\tt{}TabLabel} konstruiert.
Als Argument wird die Achsenkennung {\it{}key},
der Name der zugeh�rigen Datenachse und eine IO{-}Struktur mit
Definitionszeilen f�r die einzelnen Achsenbezeichner �bergeben.
In der Regel wird anstelle der IO{-}Struktur ein Definitionsstring
(implizite Konvertierung) �bergeben.

Jede Definitionszeile besteht aus zwei Eintr�gen (Kennung und Bezeichner),
die durch ein Leerzeichen oder einem Tabulator getrennt sind.
Die Kenneung, die keine Leerzeichen oder Tabulatoren enthalten darf,
bestimmt die Selektion in der Datenmatrix,
der Bezeichner definiert den Achsenbezeichner bei der Ausgabe
der Tabelle.
Der Bezeichner kann beliebig viele Leerzeichen und Tabulatoren enthalten.

Folgende Kennungen sind m�glich:
\begin{MPlist}
\MPitem{{\tt{}@}}
Kein Achsenbezeichner, die Komponente {\tt{}TabLabel::special} (siehe oben)
wird mit dem Bezeichner erweitert.
\MPitem{{\tt{}{\it{}name}}}
Der Name {\it{}name} gibt den zugeh�rigen Achsenbezeichner in der Datenmatrix
an.
\MPitem{{\tt{}:{\it{}name}[{\it{}liste}]}}
Der Achsenbezeichner {\it{}name} wird erst beim Laden eingerichtet und die
Daten ergeben sich durch entsprechnde Aggregation der in {\it{}list}
angef�hrten Achsenbezeichner.
\MPitem{{\tt{}.}}
Der Tabellenachse wird kein Achsenbezeichner in der Datenmatrix zugeordnet.
Bei Tabellen mit komplexeren Datentransformationen k�nnen in einer
Tabellenzeile aus einem Datenmatrixeintrag mehrere Tabellenwerte
errechnet werden. Die Tabellenachse ben�tigt damit mehr Spalten
als Datenwerte vorhanden sind.
Bei Zeilenachsen k�nnen damit benannte Leerzeilen konstruiert werden.
\end{MPlist}
Zu Beginn eines Bezeichners k�nnen spezielle Kennungen stehen,
die der Reihe nach abgefragt werden. Das erste Zeichen, das
keiner Kennung mehr entspricht, stellt den Beginn des Achsenbezeichners
dar.

Folgende Kennungen sind m�glich:
\begin{MPlist}
\MPitem{{\tt{}[{\it{}size}]}}
Explizite Angabe der H�he einer Zeile, bzw. Breite einer Spalte.
Nur notwendig, wenn nicht alle Zeilen die gleiche H�he oder alle
Spalten die gleiche Breite haben.
\MPitem{{\tt{}@}}
Eine Leerzeile wird ausgegeben (Alte Definition,
Besser: Eigene Definitionszeile mit Punkt als Kennung und leeren
Bezeichner, siehe 1. Beispiel).
\MPitem{{\tt{}$\ast$}}
Die Zeile wird fett gesetzt.
\MPitem{{\tt{}{-}}}
Die Zeile wird mit fetten Linien gesetzt.
\MPitem{{\tt{}+}}
Die Zeile wird einger�ckt. Mehrere Einr�ckungen sind zul�ssig.
\end{MPlist}
Mit Ausnahme der Spaltenbreite werden alle anderen Kennungen f�r
Spaltenachsen ignoriert.

Anstelle der Verwendung einer IO{-}Struktur bei der Konstruktion
einer Tabellenachse kann auch die Funktion {\tt{}TabLabel::entry}
verwendet werden. Als Argument kann

Anstelle des Konstruktors mit einer IO{-}Struktur der 
\MPsection{BEISPIELE}
Das folgende Beispiel zeigt eine einfache Zeilenachse mit einer Gliederung
durch Leerzeilen.

\begin{MPnofill}
TabLabel({\tt\symbol{34}}sv{\tt\symbol{34}}, {\tt\symbol{34}}sv{\tt\symbol{34}}, string !
MANN    M�nner
FRAU    Frauen
.
A00{-}18  Bis 18
A19{-}24  19 bis 24
A25{-}29  25 bis 29
A30{-}39  30 bis 39
A40{-}49  40 bis 49
A50{-}54  50 bis 54
A55{-}59  55 bis 59
A60{-}99  60 und �lter
.
Burg    Burgenland
Kaer    K�rnten
Nied    Nieder�sterreich
Ober    Ober�sterreich
Salz    Salzburg
Stei    Steiermark
Tiro    Tirol
Vora    Vorarlberg
Wien    Wien
Blnz    Keine Zuordnung
.
INL     Inl�nder$\backslash$Innen
AUS     Ausl�nder$\backslash$Innen
.
GES     $\ast$Insgesamt
!);
\end{MPnofill}
Der folgende Spaltenkopf besitzt eine Doppelgliederung.
Die Achsenbezeichner beziehen sich auf die unterste Gliederung.
Beachte, da� zwischen der letzten Sonderdefinition und dem
ersten Achsenbezeichner automatisch ein {\tt\symbol{34}}$\backslash$norule{\tt\symbol{34}} eingef�gt
wird.

\begin{MPnofill}
TabLabel({\tt\symbol{34}}hdr8{\tt\symbol{34}}, {\tt\symbol{34}}epi{\tt\symbol{34}}, string !
@       $\backslash$$\backslash$lineheight\{1\}
@       $\backslash$$\backslash$norule \& Unselbst�ndig Erwerbst�tige$\backslash$$\backslash$fn1 \&
@       $\backslash$$\backslash$lmulticol\{3\}\{davon haben mindestens eine Episode von:\}$\backslash$$\backslash$cr 
@       $\backslash$$\backslash$lineheight\{1\}
GES     $\backslash$$\backslash$norule
AEPI    Arbeitslosigkeit 
PEPI    Pension 
SEPI    Sonstiges$\backslash$$\backslash$fn2
!);
\end{MPnofill}
Die Dummyachsen CDummyTabLabel und LDummyTabLabel sind folgend definiert:

\begin{MPnofill}
TabLabel CDummyTabLabel = TabLabel({\tt\symbol{34}}CDummy{\tt\symbol{34}}, NULL, string !
A       Spalte A
B       Spalte B
C       Spalte C
D       Spalte D
!);
~
TabLabel LDummyTabLabel = TabLabel({\tt\symbol{34}}LDummy{\tt\symbol{34}}, NULL, string !
1       Zeile 1
2       Zeile 2
3       Zeile 3
4       Zeile 4
.       .......
n       $\ast$Zeile n
!);
\end{MPnofill}
\MPsection{SIEHE AUCH}
texmerge(1), SynTab(7).
