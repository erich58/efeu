% German Stilfile mit ISO-Latin1 Sonderzeichen

\input {german.sty}

%	Absicherung gegen mehrfache Einbindung

\expandafter\ifx\csname Latin@active\endcsname\relax
\else \expandafter\endinput \fi
 
\def\Latin@active{%
	\catcode196=\active
	\catcode214=\active
	\catcode220=\active
	\catcode228=\active
	\catcode246=\active
	\catcode252=\active
	\catcode223=\active
}

\def\Latin@letter{%
	\catcode196=11
	\catcode214=11
	\catcode220=11
	\catcode228=11
	\catcode246=11
	\catcode252=11
	\catcode223=11
}

\Latin@active

\def�{\protect"A}
\def�{\protect"O}
\def�{\protect"U}
\def�{\protect"a}
\def�{\protect"o}
\def�{\protect"u}
\def�{\protect"s}

%	Indexhilfsmakros

\def\LatinIndex{%
	\def�{\dq\protect"A}%
	\def�{\dq\protect"O}%
	\def�{\dq\protect"U}%
	\def�{\dq\protect"a}%
	\def�{\dq\protect"o}%
	\def�{\dq\protect"u}%
	\def�{\dq\protect"s}%
}

\def\LatinSort{%
	\def�{AE}%
	\def�{OE}%
	\def�{UE}%
	\def�{ae}%
	\def�{oe}%
	\def�{ue}%
	\def�{ss}%
}

\newcommand{\gindex}[1]{%
	{\LatinSort\xdef\@tempA{#1}}%
	\def\ISO@tempB{#1}%
	{\LatinIndex\index{\@tempA @#1}}{#1}}
